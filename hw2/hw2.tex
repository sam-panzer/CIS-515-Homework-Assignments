\documentclass{article}
\usepackage{fullpage}
\usepackage{amsmath}

\setcounter{secnumdepth}{0}
\newcommand{\emptyspace}{\{0\}}
\begin{document}

\begin{center}CIS 515 --- HW2\\Sam Panzer and Kevin Shi\end{center}
\subsection{Problem B1}
We'll do this by induction on $n$.
The base case is trivial, since then $AB := A_1(B_1^\top)^\top$.

For the inductive step, we assume that $n >= 1$, and we let $A'$ be the first
$n - 1$ columns of $A$ and $B'$ be the first $n - 1$ rows of $B$. Furthermore,
let $a$ be the last column of $A$ and $b$ be the last column of $B$.
Our induction hypothesis is that
$A'B' = A_1(B_1'^\top)^\top + \dots + A_{n-1}(B_{n-1}'^\top)^\top $.
Note that
\[(AB)_{i,j} = \left( \sum_{k=1}^{n-1} A_{i,k}B_{k,j} \right) + a_ib_j
             = (A'B')_{i,j} + a_ib_j\]
Because the sum is just $A'B'$, we can break this up to be
\[AB = A'B' + ab = A_1(B_1'^\top)^\top + \dots + A_{n-1}(B_{n-1}'^\top)^\top  +
ab\]
Since $ab$ is defined to be $A_n(B_n^\top)^\top$, we just cite the induction
hypothesis to conclude the proof.

\subsection{Problem B2}
$f: E \rightarrow F$ is a bijective lineasr map. To show that $f^{-1}$ is
linear, note that $f^{-1} \circ f = id$ and consider
\begin{eqnarray*}
f^{-1}(a) + f^{-1}(b) &=& id(f^{-1}(a) + f^{-1}(b) )\\
&=& f^{-1}(f(f^{-1}(a) + f^{-1}(b))) \\
&=&f^{-1}(f(f^{-1}(a)) + f(f^{-1}(b)))\\
&=& f^{-1}(a + b)
\end{eqnarray*}
\subsection{Problem B3}
\subsubsection{(1)}
This proof requires two directions.
First, let's assume that $z = (u_1,\dots,u_{i-1},v,u_{i+1}, \dots, u_p) \in Z_i$
for some $i$.
This means that 
\[a(z) = \sum_{j = 1, j \neq i}^p u_j + v 
      = \sum_{j = 1, j \neq i}^p u_j - \sum_{j = 1, j \neq i}^p u_j = 0\]
Thus $a(z) = 0$ for each $z$ in any $Z_i$. $a$'s linearity therefore implies
that any linear combination of vectors drawn from the $Z_i$ is therefore in its
kernel, which means that $\ker(a) \supseteq Z_1 + \dots + Z_p$.

For the other direction, we assume that $u = (u_1,\dots,u_p) \in \ker(a)$.
Then $u_i = -\sum_{j = 1, j \neq i} u_j$ for each $i$, since $a(u) = 0$ implies
that $u_1 + \dots + u_p = 0$.
This means that $u$ satisfies the criterion for membership in any $Z_i$, so it
must be a member of $Z_ + \dots + Z_p$.
Therefore $Z_ + \dots + Z_p \supseteq \ker(a)$, meaning that the two spaces are
equivalent.

\subsubsection{(2)}
The sum is a direct sum iff $a$ is injective.
First, let's assume that $a$ is in fact injective, which means that 
$u_1 + \dots + u_p = v_1 + \dots + v_p$ iff $u_i = p_i \forall i$.
We need to show that
\[V = U_i \cap \left( \sum_{j=1,j\neq i}^p U_j \right)\]
is the trivial vector space. We suppose that $x \in V,$ which means that
$x \in U_i$ and $x = \displaystyle{\sum_{j=1,j \neq i}^p}\lambda_j u_j$ where
$u_j \in U_j$.

\subsection{Problem B4}

\end{document}
